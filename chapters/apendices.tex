\chapter{Apéndices}
  \label{Apendices}

  \section{Apéndice A}
    \label{Apendices:ApendiceA}
      El \emph{análisis de inventario de servicios} implica definir un grupo de entidades y procesos de negocio que permitan establecer una estructura básica para los servicios contenidos dentro del inventario (\emph{service blueprints}), de modo que estos no se superpongan entre sí y trabajen en conjunto de forma óptima para los procesos que abarca dicho inventario. El éxito de esta etapa requiere de conocimiento sobre el y/o los negocios de la organización.

      Un \emph{análisis orientado a servicio} es la primer etapa en la definición de servicios individuales. Esta etapa contribuye a los service blueprints y tiene como objetivo la identificación de distintos tipos de servicios dentro de el o los inventarios definidos.

      El \emph{diseño orientado a servicio} es un paso más hacia la definición de cada servicio; en esta etapa se definen los contratos de software que deberán cumplir los mismos. Estos son elaborados entre analistas de negocios y arquitectos.

      Una vez obtenido el contrato, se pasa a elaborar el \emph{diseño de la lógica de servicio} para su posterior \emph{desarrollo} y \emph{verificación}, los cuales requieren de arquitectos, analistas de negocios, desarrolladores y especialistas en aseguramiento de la calidad, roles que deben estar especializados en el área de negocio del organismo.

      \emph{En desarrollo}
