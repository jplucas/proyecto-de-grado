\section{Calidad}
\label{Solucion:Calidad}

Tras el análisis de la realidad en AGESIC visto en el capitulo \ref{Analisis}, en esta sección se propondrá un modelo de calidad para monitorear los servicios de AGESIC.

El modelo esta basado en el metamodelo abstracto de calidad analizado en el capitulo \ref{MarcoConceptual}.
La construcción del modelo de calidad propuesto esta basado a través de revelamiento con el cliente. Una vez establecido los datos prioritarios se definen Dimensiones y Factores como se proponen en el capitulo \ref{MarcoConceptual:metamodelo_calidad}.
Para lograr estas definiciones, se realizo una comparación con los principales atributos de calidad de los modelos analizados en la sección \ref{MarcoConceptual:modelos_calidad} que son de interés para AGESIC.
  
   \begin{table}[h]
    \begin{tabular}{ |c | c | c  | c|} 
    \hline
     \textbf{Atributos} & \textbf{OASIS} & \textbf{S-Cube}  & \textbf{IBM}  \\
      \hline
     	\emph{Response Time} & 
		\includegraphics[scale=0.5]{icon_5801} & 
		\includegraphics[scale=0.5]{icon_5801} & \\
	 \hline
     	\emph{Scalability} & 
		\includegraphics[scale=0.5]{icon_5801} & 
		\includegraphics[scale=0.5]{icon_5801} & 
		\includegraphics[scale=0.5]{icon_5801} \\
	\hline
     	\emph{Throughput} & 
		\includegraphics[scale=0.5]{icon_5801} & 
		\includegraphics[scale=0.5]{icon_5801} & 
		\includegraphics[scale=0.5]{icon_5801} \\
	\hline
     	\emph{Latency} & 
		\includegraphics[scale=0.5]{icon_5801} & 
		\includegraphics[scale=0.5]{icon_5801} & 
		\includegraphics[scale=0.5]{icon_5801} \\
	\hline
     	\emph{Reliability} & 
		\includegraphics[scale=0.5]{icon_5801} & 
		\includegraphics[scale=0.5]{icon_5801} & 
		\includegraphics[scale=0.5]{icon_5801} \\
	\hline
     	\emph{Regulatory} & 
		\includegraphics[scale=0.5]{icon_5801} & 
		\includegraphics[scale=0.5]{icon_5801} & 
		\includegraphics[scale=0.5]{icon_5801} \\
	\hline
     	\emph{Availability} & 
		\includegraphics[scale=0.5]{icon_5801} & 
		\includegraphics[scale=0.5]{icon_5801} & 
		\includegraphics[scale=0.5]{icon_5801} \\
	\hline
     	\emph{Accessibility} & 
		\includegraphics[scale=0.5]{icon_5801} & 
		\includegraphics[scale=0.5]{icon_5801} & \\
	\hline
     	\emph{Successability} & 
		\includegraphics[scale=0.5]{icon_5801} & 
		\includegraphics[scale=0.5]{icon_5801} & \\
	\hline
     	\emph{Time To Repair} & &  & 
		\includegraphics[scale=0.5]{icon_5801} \\
	\hline
     	\emph{Transaction Time} & &
     	\includegraphics[scale=0.5]{icon_5801}  &  \\
	\hline
	\hline
     	\emph{Service Cost} & 
		\includegraphics[scale=0.5]{icon_5801} & 
		\includegraphics[scale=0.5]{icon_5801} & \\
	\hline
     	\emph{Usability} & 
		\includegraphics[scale=0.5]{icon_5801} & 
		\includegraphics[scale=0.5]{icon_5801} & \\	
	\hline
     	\emph{Satisfaction} & 
		\includegraphics[scale=0.5]{icon_5801} & 
		\includegraphics[scale=0.5]{icon_5801} & \\		
	\hline
     	\emph{Content accessibility} & & 
		\includegraphics[scale=0.5]{icon_5801} & \\
	\hline
	\hline
     	\emph{Confidentiality} & 
		\includegraphics[scale=0.5]{icon_5801} & 
		\includegraphics[scale=0.5]{icon_5801} & 
		\includegraphics[scale=0.5]{icon_5801} \\
	\hline
     	\emph{Integrity} & 
		\includegraphics[scale=0.5]{icon_5801} & 
		\includegraphics[scale=0.5]{icon_5801} & 
		\includegraphics[scale=0.5]{icon_5801} \\
	\hline
     	\emph{Authentication} & 
		\includegraphics[scale=0.5]{icon_5801} & 
		\includegraphics[scale=0.5]{icon_5801} & 
		\includegraphics[scale=0.5]{icon_5801} \\
	\hline
     	\emph{Access Control} & 
		\includegraphics[scale=0.5]{icon_5801} & 
		\includegraphics[scale=0.5]{icon_5801} & 
		\includegraphics[scale=0.5]{icon_5801} \\
	\hline
     	\emph{Non-Repudiation} & 
		\includegraphics[scale=0.5]{icon_5801} & 
		\includegraphics[scale=0.5]{icon_5801} & 
		\includegraphics[scale=0.5]{icon_5801} \\
	\hline    
     	\emph{Traceability} & 
		\includegraphics[scale=0.5]{icon_5801} & 
		\includegraphics[scale=0.5]{icon_5801} & 
		\includegraphics[scale=0.5]{icon_5801} \\	
	\hline
	\hline	
	  	\emph{Data validity} &  & 
		\includegraphics[scale=0.5]{icon_5801} &  \\	
	\hline	  
	 	\emph{Data policy} &  & 
		\includegraphics[scale=0.5]{icon_5801} &  \\
	\hline	
	  	\emph{Data integrity} &  & 
		\includegraphics[scale=0.5]{icon_5801} &  \\
	\hline	
	   \emph{Data encryption Non-repudation} &  & 
		\includegraphics[scale=0.5]{icon_5801} &  \\	
	\hline
	\hline
     	\emph{WS Interoperability} & 
		\includegraphics[scale=0.5]{icon_5801} & 
		\includegraphics[scale=0.5]{icon_5801} &  \\	
	 \hline
    \end{tabular}
    \caption{Comparación de atributos de calidad con los modelos OASIS,S-Cube y IBM}
    \label{tabla:comparacion_atributos}
  \end{table}
  
En la tabla \ref{tabla:comparacion_atributos} se puede observar como los atributos se relacionan entre los modelos analizados. Gran cantidad de atributos están relacionados entre las distintas propuestas. En base a este análisis reagrupamos los conceptos en común, dando como resultado el conjunto de dimensiones y factores que se presenta en la figura \ref{figura:sol_mod_calidad}.
  \begin{figure}[h]
    \centering
    \includegraphics[scale=0.5]{sol_modelo_calidad}
    \caption{Propuesta de Modelo de Calidad:Dimensiones y Factores}
    \label{figura:sol_mod_calidad}
  \end{figure}

\subsubsection{Descripción de las dimensiones}
La solución de modelo de calidad presentado en la figura \ref{figura:sol_mod_calidad} consta de las siguientes dimensiones: 
		\begin{enumerate}
			\item \emph{Security (Seguridad)} es la dimensión enfocada a monitorear la calidad de servicios Web Services en aspectos de seguridad, de forma de poder analizar diferentes controles que permitan proteger al servicios.
			\item  \emph{Performance and Stability (Performances y Estabilidad)} es la dimensión enfocada al análisis del rendimiento de los servicios y su comportamiento a lo largo del tiempo.
			\item  \emph{Usability and Cost (Usabilidad y Costos)} es la dimensión que permite analizar los aspectos que aportan a la facilidad,aprendizaje y conformidad por parte de los usuarios que consumen los servicios , junto con los distintos costos que puede tener los mismos.
			\item  \emph{Interoperability (Interoperabilidad)} es la dimensión que permite evaluar cuanto se puede adaptar el servicio a los estándares de organizaciones reconocidas como WS-I.
		\end{enumerate}


\subsubsection{Descripción de los factores}
En el capitulo \ref{MarcoConceptual} vemos que una dimension se pude ver como una agrupación de factores de calidad con el mismo propósito. A continuación se definen los factores por dimensión que forman parte de la solución del modelo de calidad presentado.
\begin{enumerate}
\item \emph{Security}
	\begin{enumerate}
		\item \emph{Confidentiality (Confidencialidad)} es la capacidad de prevenir qué usuarios no autorizados vean o accedan al servicio/mensajes. Usa control de acceso y encriptación.
		\item \emph{Integrity (Integridad)}  es la capacidad de proteger servicios/mensajes contra modificación,eliminación o creación. Integridad basada en mensaje o basada en transporte.
		\item \emph{Authentication (Autenticación)} verificar que un objeto es confiable para la transmisión.
		\item \emph{Access Control (Control de Acceso)} verifica el control sobre el acceso a servicio/mensajes para cada permiso del actor.
		\item \emph{Non-Repudiation (No Repudio)} asegurar que quien envía o recibe es quien dice ser.
		\item \emph{Traceability (Trazabilidad)} es el proceso de grabar los eventos relacionados con la seguridad en la comunicación y tomar acciones basadas en las ocurrencias de estos.
	\end{enumerate}
\item \emph{Performance and Stability}
	\begin{enumerate}
	\item \emph{Response Time (Tiempo de Respuesta)} es el factor que permite evaluar el tiempo de respuesta de los servicios.
	\item \emph{Throughput (Rendimiento)} describe el la capacidad de respuesta que tiene el servicio en resolver las solicitudes de los clientes.
	\item \emph{Availability (Disponibilidad)} es el factor que verifica si el servicio se encuentra activo para resolver solicitudes.
	\item \emph{Accessibility (Accesibilidad)} permite saber si el servicio se encuentra activo en la plataforma para ser consumido por los clientes.
	\item \emph{Successability (Grado de Éxito)} permite analizar la capacidad de éxito del servicio para resolver solicitudes de los clientes de forma correcta.
	\end{enumerate}
\item \emph{Usability and Cost}
	\begin{enumerate} 
	\item \emph{Service Cost (Costos de servicio)} es el factor que analiza los costos que aplican al servicio.
	\item \emph{Usability (Usabilidad)} esta orientado al análisis de los elementos que permitan facilitar el uso de los usuarios que van a utilizar el servicio.
	\item \emph{Satisfaction (Satifación)} es el factor que evalúa el nivel de conformidad del servicio por parte de los clientes que consumen el mismo.
	\end{enumerate}
\item \emph{Interoperability}
	\begin{enumerate}
	\item \emph{Standard Adoptability (Adaptabilidad de Estándares)} verifica si el servicio cumple con estándares de interoperabilidad.
	\end{enumerate}
\end{enumerate}

\subsubsection{Definición de las métricas de calidad}
Continuando la construcción del modelo de calidad, ya establecido las dimensiones y factores ahora se establecen por que métricas se va analizar la información.
Para definir una métrica se definen establecer la semántica, unidad y granularidad de la medida \cite{Calidad:CursoCalidad}.
En la tabla \ref{tabla:definicion_metricas_per_stab} se encuentra la propuesta de métricas por factor para la dimensión \emph{Performance and Stability}. Algunas métricas tiene mas interés desde el punto de vista del cliente y desde la visión del proveedor del servicio.
Luego en la tabla \ref{tabla:definicion_metricas_wsi_usa_cost} se encuentran las métricas de las dimensiones  \emph{sability and Cost} e \emph{Interoperability}.
Finalmente en la tabla \ref{tabla:definicion_metricas_seguridad} están todas las métricas de la dimensión {Security}.
Como se pueden observar en las tablas mencionadas existen varias métricas que definen distintas formas de medir un mismo factor.
 \begin{table}[h]
  \centering
    \begin{tabular}{ |p{0.18\linewidth} | p{0.20\linewidth} | p{0.45\linewidth} | p{0.15\linewidth} | p{0.20\linewidth}|} 
    \hline
     \textbf{Factor} & \textbf{Métrica} & \textbf{Semántica}  & \textbf{Unidad}  & \textbf{Granularidad}  \\
      \hline
       \hline
       \emph{Response Time} & \emph{ Average Response Time (Client)} &Tiempo promedio en el cual resuelve una solicitud al servicio invocado. Es el tiempo que espera el cliente & Segundos & Rango de tiempo\\
      \hline
       \emph{ Response Time}&  \emph{Max Response Time (Client) }& Máximo tiempo que llevo resolver una solicitud al servicio invocado  & Segundos & Rango de tiempo\\
      \hline
       \emph{Response Time} &  \emph{Min Response Time (Client)}  & Máximo tiempo que llevo resolver una solicitud al servicio invocado & Segundos & Rango de tiempo\\
      \hline
       \emph{Response Time} &  \emph{Average Response Time (Provider)} &Tiempo promedio en el cual resuelve una solicitud al servicio invocado sin tener en cuenta los tiempo de latencia que puede sufrir el servicio (ejemplo transformaciones). & Segundos & Rango de tiempo\\
      \hline
      \emph{ Response Time}&  \emph{ Max Response Time (Provider)}  & Máximo tiempo que llevo resolver una solicitud al servicio invocado sin tener en cuenta los tiempo de latencia que puede sufrir el servicio  & Segundos & Rango de tiempo\\
      \hline
       \emph{Response Time} &  \emph{Min Response Time (Provider)}  & Máximo tiempo que llevo resolver una solicitud al servicio invocado sin tener en cuenta los tiempo de latencia que puede sufrir el servicio & Segundos & Rango de tiempo\\
      \hline
        \emph{Throughput} &  \emph{Throughput (Client)}  & Cantidad de solicitados al servicio, por segundo puede resolver & Cantidad transacciones/Seg. & Rango de tiempo\\
      \hline
       \emph{Throughput} &  \emph{Throughput (Provider)}  & Cantidad de solicitados al servicio que resuelve al endpoint del proveedor, por segundo & Cantidad transacciones/Seg. & Rango de tiempo\\
      \hline
        \emph{Availability} &  \emph{Availability}  & Proporción de tiempo en que el servidor de Web Services está en funcionamiento & Numérico & Rango de tiempo\\
      \hline
        \emph{Accessibility} &  \emph{Accessibility}  & Proporción de tiempo en que el se invoca al Web Services y se encuentre disponible para ser invocado & Numérico & Rango de tiempo\\
       \hline
        \emph{Successability} &  \emph{Successability} & Proporción de tiempo en que el servicio es invocado y recibe una respuesta sin error por parte del proveedor del mismo. & Numérico & Rango de tiempo\\
      \hline
    \end{tabular}
    \caption{Métricas de la dimensión Performance and Stability}
    \label{tabla:definicion_metricas_per_stab}
  \end{table}

 \begin{table}[h]
  \centering
    \begin{tabular}{ |p{0.18\linewidth} | p{0.20\linewidth} | p{0.45\linewidth} | p{0.15\linewidth} | p{0.20\linewidth}|} 
    \hline
     \textbf{Factor} & \textbf{Métrica} & \textbf{Semántica}  & \textbf{Unidad}  & \textbf{Granularidad}  \\
      \hline
      \hline
       \emph{Service Cost} &  \emph{Price} & Indica si el servicio tiene costo & Booleano & Servicio \\
       \hline
       \emph{Service Cost} &  \emph{Cost for Transaction} & Indica a partir de que cantidad de solicitudes al servicio, se le debe cobrar al cliente  & Numérico & Transacciones \\
       \hline
       \emph{Service Cost} &  \emph{Penalty} & Precio de sanciones al proveedor por incumplimiento de acuerdos de calidad del servicio (SLA)& Numérico & Servicio \\
        \hline
       \emph{Usability} & {Documentation Services} & Indica si el servicio dispone de una buena documentación que permita entender e invocar el servicio & Booleano & Servicio \\
      \hline
       \hline
        \emph{Standard Adoptability} &  \emph{WS-I Basic Profile} & Indica si el servicio cumple con el estándar de perfiles definidos por \emph{Web Services Interoperability Organization} & Boolean & Servicio \\
       \hline
       \hline
    \end{tabular}
    \caption{Métricas de las dimensiones Interoperability y Usability and Cost }
    \label{tabla:definicion_metricas_wsi_usa_cost}
  \end{table}

 \begin{table}[h]
  \centering
    \begin{tabular}{ |p{0.18\linewidth} | p{0.20\linewidth} | p{0.45\linewidth} | p{0.15\linewidth} | p{0.20\linewidth}|} 
    \hline
     \textbf{Factor} & \textbf{Métrica} & \textbf{Semántica}  & \textbf{Unidad}  & \textbf{Granularidad}  \\
      \hline
       \hline
       \emph{Confidentiality} &  \emph{Confidentiality} & El servicio debe cumplir con la propiedad de seguridad de confidencialidad del mismo & Boolean & Servicio \\
       \hline
       \emph{Integrity} &  \emph{Integrity} & El servicio debe cumplir con la propiedad de seguridad de integridad del mismo & Boolean & Servicio \\
       \hline
       \emph{Authentication} &  \emph{Authentication} & El servicio debe cumplir con la propiedad de seguridad de autenticación del mismo & Boolean & Servicio \\
       \hline
       \emph{Access Control} &  \emph{Access Control} & El servicio debe cumplir con la propiedad de seguridad de control de acceso del mismo & Boolean & Servicio \\
       \hline
       \emph{Non-Repudiation} &  \emph{Non-Repudiation} &El servicio debe cumplir con la propiedad de seguridad de no repudio del mismo & Boolean & Servicio \\
        \hline
   \emph{Traceability} &  \emph{Traceability} & El servicio debe cumplir con la propiedad de seguridad de trazabilidad del mismo  & Boolean & Servicio \\
      \hline
    \end{tabular}
    \caption{Métricas de la dimensión Seguridad}
    \label{tabla:definicion_metricas_seguridad}
  \end{table}

\subsubsection{Definición de los métodos de las métricas de calidad}
Finalmente se encuentra la ultima etapa del modelo de calidad donde se establece como medir las métricas de calidad. La implementación del método depende de la aplicación en concreto \cite{Calidad:CursoCalidad}. Por lo que dependiendo de la aplicación, se tendrán que establecer las operaciones que permitan calcular las métricas del modelo. En el capitulo \ref{Implementacion} se analiza la forma de implementar las métricas en base a un ESB especifico.